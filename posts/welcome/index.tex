% Options for packages loaded elsewhere
\PassOptionsToPackage{unicode}{hyperref}
\PassOptionsToPackage{hyphens}{url}
\PassOptionsToPackage{dvipsnames,svgnames,x11names}{xcolor}
%
\documentclass[
  letterpaper,
  DIV=11,
  numbers=noendperiod]{scrartcl}

\usepackage{amsmath,amssymb}
\usepackage{iftex}
\ifPDFTeX
  \usepackage[T1]{fontenc}
  \usepackage[utf8]{inputenc}
  \usepackage{textcomp} % provide euro and other symbols
\else % if luatex or xetex
  \usepackage{unicode-math}
  \defaultfontfeatures{Scale=MatchLowercase}
  \defaultfontfeatures[\rmfamily]{Ligatures=TeX,Scale=1}
\fi
\usepackage{lmodern}
\ifPDFTeX\else  
    % xetex/luatex font selection
\fi
% Use upquote if available, for straight quotes in verbatim environments
\IfFileExists{upquote.sty}{\usepackage{upquote}}{}
\IfFileExists{microtype.sty}{% use microtype if available
  \usepackage[]{microtype}
  \UseMicrotypeSet[protrusion]{basicmath} % disable protrusion for tt fonts
}{}
\makeatletter
\@ifundefined{KOMAClassName}{% if non-KOMA class
  \IfFileExists{parskip.sty}{%
    \usepackage{parskip}
  }{% else
    \setlength{\parindent}{0pt}
    \setlength{\parskip}{6pt plus 2pt minus 1pt}}
}{% if KOMA class
  \KOMAoptions{parskip=half}}
\makeatother
\usepackage{xcolor}
\setlength{\emergencystretch}{3em} % prevent overfull lines
\setcounter{secnumdepth}{-\maxdimen} % remove section numbering
% Make \paragraph and \subparagraph free-standing
\ifx\paragraph\undefined\else
  \let\oldparagraph\paragraph
  \renewcommand{\paragraph}[1]{\oldparagraph{#1}\mbox{}}
\fi
\ifx\subparagraph\undefined\else
  \let\oldsubparagraph\subparagraph
  \renewcommand{\subparagraph}[1]{\oldsubparagraph{#1}\mbox{}}
\fi


\providecommand{\tightlist}{%
  \setlength{\itemsep}{0pt}\setlength{\parskip}{0pt}}\usepackage{longtable,booktabs,array}
\usepackage{calc} % for calculating minipage widths
% Correct order of tables after \paragraph or \subparagraph
\usepackage{etoolbox}
\makeatletter
\patchcmd\longtable{\par}{\if@noskipsec\mbox{}\fi\par}{}{}
\makeatother
% Allow footnotes in longtable head/foot
\IfFileExists{footnotehyper.sty}{\usepackage{footnotehyper}}{\usepackage{footnote}}
\makesavenoteenv{longtable}
\usepackage{graphicx}
\makeatletter
\def\maxwidth{\ifdim\Gin@nat@width>\linewidth\linewidth\else\Gin@nat@width\fi}
\def\maxheight{\ifdim\Gin@nat@height>\textheight\textheight\else\Gin@nat@height\fi}
\makeatother
% Scale images if necessary, so that they will not overflow the page
% margins by default, and it is still possible to overwrite the defaults
% using explicit options in \includegraphics[width, height, ...]{}
\setkeys{Gin}{width=\maxwidth,height=\maxheight,keepaspectratio}
% Set default figure placement to htbp
\makeatletter
\def\fps@figure{htbp}
\makeatother

\KOMAoption{captions}{tableheading}
\makeatletter
\makeatother
\makeatletter
\makeatother
\makeatletter
\@ifpackageloaded{caption}{}{\usepackage{caption}}
\AtBeginDocument{%
\ifdefined\contentsname
  \renewcommand*\contentsname{Table of contents}
\else
  \newcommand\contentsname{Table of contents}
\fi
\ifdefined\listfigurename
  \renewcommand*\listfigurename{List of Figures}
\else
  \newcommand\listfigurename{List of Figures}
\fi
\ifdefined\listtablename
  \renewcommand*\listtablename{List of Tables}
\else
  \newcommand\listtablename{List of Tables}
\fi
\ifdefined\figurename
  \renewcommand*\figurename{Figure}
\else
  \newcommand\figurename{Figure}
\fi
\ifdefined\tablename
  \renewcommand*\tablename{Table}
\else
  \newcommand\tablename{Table}
\fi
}
\@ifpackageloaded{float}{}{\usepackage{float}}
\floatstyle{ruled}
\@ifundefined{c@chapter}{\newfloat{codelisting}{h}{lop}}{\newfloat{codelisting}{h}{lop}[chapter]}
\floatname{codelisting}{Listing}
\newcommand*\listoflistings{\listof{codelisting}{List of Listings}}
\makeatother
\makeatletter
\@ifpackageloaded{caption}{}{\usepackage{caption}}
\@ifpackageloaded{subcaption}{}{\usepackage{subcaption}}
\makeatother
\makeatletter
\@ifpackageloaded{tcolorbox}{}{\usepackage[skins,breakable]{tcolorbox}}
\makeatother
\makeatletter
\@ifundefined{shadecolor}{\definecolor{shadecolor}{rgb}{.97, .97, .97}}
\makeatother
\makeatletter
\makeatother
\makeatletter
\makeatother
\ifLuaTeX
  \usepackage{selnolig}  % disable illegal ligatures
\fi
\IfFileExists{bookmark.sty}{\usepackage{bookmark}}{\usepackage{hyperref}}
\IfFileExists{xurl.sty}{\usepackage{xurl}}{} % add URL line breaks if available
\urlstyle{same} % disable monospaced font for URLs
\hypersetup{
  pdftitle={Introducing Deep Learning Revision Research Blog},
  pdfauthor={Jean de Dieu Nyandwi},
  colorlinks=true,
  linkcolor={blue},
  filecolor={Maroon},
  citecolor={Blue},
  urlcolor={Blue},
  pdfcreator={LaTeX via pandoc}}

\title{Introducing Deep Learning Revision Research Blog}
\author{Jean de Dieu Nyandwi}
\date{2023-07-15}

\begin{document}
\maketitle
\ifdefined\Shaded\renewenvironment{Shaded}{\begin{tcolorbox}[frame hidden, interior hidden, breakable, sharp corners, borderline west={3pt}{0pt}{shadecolor}, boxrule=0pt, enhanced]}{\end{tcolorbox}}\fi

Greetings,

I am super delighted to introduce Deep Learning Revision research blog.
The aim of Deep Learning Revision blog is to elucidate fundamentals and
novel techniques in AI research.

There has been a cambrian explosion in AI research in recent years.
Since 2012 when AlexNet showed significant performance on image
recognition, deep learning has profoundly reshaped the field of AI. In
fact, in the year 2010-2020, there were developments of new models,
better optimization techniques, evaluation benchmarks, etc\ldots We have
seen models like Transformers taking over natural language
processing(NLP), computer vision, and at the same time showing potential
in complex problems such as in robotics, reinforncement learning,
etc\ldots It's even unbelievable what happened in the last two years.
For instance, we have witnessed AI systems that are capable of
generating photorealistic images, transcribing speeches with high
accuracy, understanding multiple modalities, and generating realistic
texts. Let's expand that and provide a few specific examples:

\begin{itemize}
\item
  Image generation is arguably one of the fields that have had massive
  breakthroughts in last two years. In 2017, at best, the images you
  could generate were 32x32 pixels filled up with too much blogs. Fast
  forward, in 2023, image generation systems have improved to the extent
  it's hard to differentiate real and fake images, and in the future, it
  will be the same for videos as well, if not already. Examples of
  seminal works in image generation are DALLE•2, Stable Diffusion,
  Imagen, among others.
\item
  Text generation has had many breakthroughs as image generation in
  recent times. Large language models have taken the world by a storm.
  The biggest revolution has mostly been in natural language interfaces
  like ChatGPT and Google Bard. Natural language interfaces are powered
  by large language models pretrained on massive amount of text data.
  Example of large language models are GPT-3, GPT-4, PaLM, PaLM-2,
  LLaMA, LLaMA2, Chinchilla, BLOOM, among others.
\item
  Multimodal learning is another concrete example of recent advancements
  in deep learning. Designing single systems that can see and hear
  what's around and respond accordingly is the holy grail of AI.
  Although there are still challenges, it is inaruagable that the AI
  research community has solved independent modalities to a large
  extent. Agents in real-world however must have the ability to learn
  from multiple modalities jointly. A challenge now is thus how to
  design AI systems that can efficiently extract meaningful
  representations from different modalities without using modality
  specific encodings. There has been many remarkable works in multimodal
  learning(most of them are surprisingly visual language models).
  Notable examples are Flamingo, Gato, BLIP(and BLIP-2), GIT, GPT-4,
  PaLI and PaLI-X, among others. With the advancements of visual
  recognition models and language models as generalization engine{[}Eric
  vang blog{]}, I expect to see massive breakthroughts in this area in
  the next months.
\item
  Robotics is another field that is yet to be influenced by deep
  learning. Robotic tasks typically require low-level engineering and
  there are so much potential if modern deep learning algorithms can
  take care of those low-level tasks by learning from massive amount of
  data. Robotics as a field poses many challenges, but there are many
  ongoing works around deep robot learning and this also the field that
  is going to shine in the next few years. Some notable works around
  deep robotic learning that was published recently are SayCan, Robotic
  Transformer(RT-1), VIMA, RoboCat, etc\ldots{}
\end{itemize}

I see this blog as a little corner in the universe where we can discuss
recent research, deconstruct papers, and really try to understand what's
going on. I plan to publish well studied materials across foundational
techniques and some emerging topics. This is new project and as with
other new projects, I don't have everything figured out. New articles
come at irregular time, some may take longer(I tend not to compromise
quality for speed), and some other unprojected challenges. All in all, I
am super excited for this research blog and I can't wait to publish the
first article in the next few days.

Until the first article!

Cheers!



\end{document}
